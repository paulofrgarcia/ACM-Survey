% v2-acmsmall-sample.tex, dated March 6 2012
% This is a sample file for ACM small trim journals
%
% Compilation using 'acmsmall.cls' - version 1.3 (March 2012), Aptara Inc.
% (c) 2010 Association for Computing Machinery (ACM)
%
% Questions/Suggestions/Feedback should be addressed to => "acmtexsupport@aptaracorp.com".
% Users can also go through the FAQs available on the journal's submission webpage.
%
% Steps to compile: latex, bibtex, latex latex
%
% For tracking purposes => this is v1.3 - March 2012

\documentclass[prodmode,acmcsur]{acmsmall} % Aptara syntax


\newcommand{\pgcomment}[1]{ \par\( \clubsuit\){\bf PG: }{\rm \sf #1}\(\clubsuit\) \par}
\newcommand{\rscomment}[1]{ \par\( \clubsuit\){\bf RS: }{\rm \sf #1}\(\clubsuit\) \par}


% Package to generate and customize Algorithm as per ACM style
%\usepackage[ruled]{algorithm2e}
%\renewcommand{\algorithmcfname}{ALGORITHM}
%\SetAlFnt{\small}
%\SetAlCapFnt{\small}
%\SetAlCapNameFnt{\small}
%\SetAlCapHSkip{0pt}
%\IncMargin{-\parindent}

% Metadata Information
\acmVolume{9}
\acmNumber{4}
\acmArticle{39}
\acmYear{2016}
\acmMonth{3}

% Copyright
%\setcopyright{acmcopyright}
%\setcopyright{acmlicensed}
%\setcopyright{rightsretained}
%\setcopyright{usgov}
%\setcopyright{usgovmixed}
%\setcopyright{cagov}
%\setcopyright{cagovmixed}

% DOI
\doi{0000001.0000001}

%ISSN
\issn{1234-56789}

% Document starts
\begin{document}

% Page heads
\markboth{P. Garcia et al.}{Programming Abstractions in the Reconfigurable Hardware Era}

% Title portion
\title{Programming Abstractions in the Reconfigurable Hardware Era}
\author{Paulo Garcia
\affil{Heriot-Watt University}
Robert Stewart
\affil{Heriot-Watt University}
}
% NOTE! Affiliations placed here should be for the institution where the
%       BULK of the research was done. If the author has gone to a new
%       institution, before publication, the (above) affiliation should NOT be changed.
%       The authors 'current' address may be given in the "Author's addresses:" block (below).
%       So for example, Mr. Abdelzaher, the bulk of the research was done at UIUC, and he is
%       currently affiliated with NASA.

\begin{abstract}

Programming abstractions decrease the cognitive gap between program idealization and expression. This high-level expressive power is achieved through layered abstractions - virtual machines, compilers, operating systems - which translate, at design and runtime, programmer-visible code into hardware-compatible code. While this paradigm is ideal for static, i.e., unmodifiable hardware, several problems arise when programming configurable (design time) and reconfigurable (runtime) hardware, where several abstractions break down. State of the art hardware/software co-design techniques (e.g., High Level Synthesis, Intermediate Fabrics) are, for the most part, \textit{ad hoc} patches to the traditional abstraction stack, applicable only to specific toolchains or software components.
\par In this paper, we survey current programming and hardware/software co-design abstractions. We perform a systematic analysis of how the sub-systems responsible for each abstraction are affected by hardware reconfiguration and how novel design and runtime stacks must be architected in order to thrive in the reconfigurable hardware era. We show that the paradigm in inexorably moving towards a stage where every aspect of computing - languages, compilers, interpreters, operating systems and middleware - will encompass reconfigurable hardware. 

\end{abstract}


%
% The code below should be generated by the tool at
% http://dl.acm.org/ccs.cfm
% Please copy and paste the code instead of the example below. 
%
 \begin{CCSXML}
<ccs2012>
<concept>
<concept_id>10010583.10010600.10010628</concept_id>
<concept_desc>Hardware~Reconfigurable logic and FPGAs</concept_desc>
<concept_significance>500</concept_significance>
</concept>
<concept>
<concept_id>10010583.10010682.10010684</concept_id>
<concept_desc>Hardware~High-level and register-transfer level synthesis</concept_desc>
<concept_significance>500</concept_significance>
</concept>
<concept>
<concept_id>10010583.10010682.10010689</concept_id>
<concept_desc>Hardware~Hardware description languages and compilation</concept_desc>
<concept_significance>500</concept_significance>
</concept>
<concept>
<concept_id>10010583.10010682.10010690</concept_id>
<concept_desc>Hardware~Logic synthesis</concept_desc>
<concept_significance>500</concept_significance>
</concept>
<concept>
<concept_id>10011007.10011006.10011008</concept_id>
<concept_desc>Software and its engineering~General programming languages</concept_desc>
<concept_significance>500</concept_significance>
</concept>
<concept>
<concept_id>10011007.10011006.10011060.10011062</concept_id>
<concept_desc>Software and its engineering~Architecture description languages</concept_desc>
<concept_significance>500</concept_significance>
</concept>
</ccs2012>
\end{CCSXML}

\ccsdesc[500]{Hardware~Reconfigurable logic and FPGAs}
\ccsdesc[500]{Hardware~High-level and register-transfer level synthesis}
\ccsdesc[500]{Hardware~Hardware description languages and compilation}
\ccsdesc[500]{Hardware~Logic synthesis}
\ccsdesc[500]{Software and its engineering~General programming languages}
\ccsdesc[500]{Software and its engineering~Architecture description languages}

%
% End generated code
%

% We no longer use \terms command
%\terms{Design, Algorithms, Performance}

%\keywords{Wirele}

\acmformat{Paulo~Garcia, Robert~Stewart, 2016. Programming Abstractions in the Reconfigurable Hardware Era}
% At a minimum you need to supply the author names, year and a title.
% IMPORTANT:
% Full first names whenever they are known, surname last, followed by a period.
% In the case of two authors, 'and' is placed between them.
% In the case of three or more authors, the serial comma is used, that is, all author names
% except the last one but including the penultimate author's name are followed by a comma,
% and then 'and' is placed before the final author's name.
% If only first and middle initials are known, then each initial
% is followed by a period and they are separated by a space.
% The remaining information (journal title, volume, article number, date, etc.) is 'auto-generated'.

\begin{bottomstuff}
This work is supported by 
\end{bottomstuff}

\maketitle




\input{tex/introduction}

\input{tex/codesign}



\section{Reconfigurable Runtime Systems}\label{sec:runtime}

\section{The Next Era: Fully Reconfigurable Systems}\label{sec:future}


\section{Conclusions and Future Directions}


% Acknowledgments
\begin{acks}
We acknowledge the support of the Engineering and Physical Research
Council, grant references EP/K009931/1 (Programmable embedded
platforms for remote and compute intensive image processing
applications) and EP/J015180/1 (Sensor Signal Processing).
\end{acks}

% Bibliography
\bibliographystyle{ACM-Reference-Format-Journals}
\bibliography{references}
                             % Sample .bib file with references that match those in
                             % the 'Specifications Document (V1.5)' as well containing
                             % 'legacy' bibs and bibs with 'alternate codings'.
                             % Gerry Murray - March 2012

% History dates
\received{February 2016}{March 2016}{June 2016}


\end{document}
% End of v2-acmsmall-sample.tex (March 2012) - Gerry Murray, ACM


